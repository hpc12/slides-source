{\sffamily\large\bfseries What will you learn in this class?}

In a nutshell, you will learn how to write programs that run fast and
use computers efficiently. We will start by working out how exactly a
computer does its job, and what pieces of hardware help it go faster.
By using examples, we will show you the difference between code that
goes fast and code that does not, and you will gradually learn to avoid
the latter.

We will next understand challenges in processor design that strongly
favor parallel computers--mainly restrictive economical bounds on
power clock frequency. This shifts the burden to a different part of
the machine, leading to figures such as bandwidth and latency,
encountered in memory access and communication. We will learn about
the many different machine types that have emerged, each optimized for
a different workload.  For example, a few hundred US dollars will buy
a parallel computer (a GPU) that is capable of performing, in each
second, $\sim 4\cdot 10^{12}$ floating point operations (``flops''),
but only of reading $\sim 5\cdot 10^{10}$ values from memory. We will
understand the programming of these computers and learn how to analyze
them, with the goal of figuring out if a given computer is a good fit
for a target workload.

Here's a rough syllabus of what we will cover:
\vspace{-1em}
\begin{itemize}
\setlength{\itemsep}{-1mm}
  \item Basic processor architecture\\
    Performance of sequential code
  \item Why go parallel? Forms of parallelism
  \item Shared Memory and OpenMP
  \item Tools and Debuggers
  \item GPUs and OpenCL
  \item Distributed Memory and MPI
  \item Common Patterns in Parallel Algorithms
  \item Partitioning and Load Balancing
\end{itemize}


{\sffamily\large\bfseries Who should take this class?}

We welcome students (and researchers!) from all departments at NYU.
The class is taught at a level suitable for advanced undergraduate and
entering graduate students. If you have a computational need from an
existing project, even better! We will try to help you work on your
project as part of the course.
To get the most out of the course, you should be a proficient C
programmer.

We will provide problem sets that accompany the class and reinforce
the learned material each week. As with each programming-related
class, actually writing code is the main way of learning, so these
assignments are an integral part of the course. We will provide
access to large parallel machines where you can try out what we taught
you.

If you are taking the class for credit, you will also do a more
ambitious final project towards the end of the class.

We're looking forward to seeing you in the fall!

\hfill \emph{Marsha Berger} \hfill \emph{Andreas Klöckner}
