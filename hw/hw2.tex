\documentclass[11pt]{article}

\usepackage[letterpaper,margin=1in]{geometry}

\usepackage{akteach}
\usepackage{amsmath}
\usepackage{amssymb}
\usepackage{enumerate}
\usepackage{keystroke}

\usepackage{natbib}
\newcommand{\doi}[1]{\href{http://dx.doi.org/#1}{doi: #1}}

\lstset{
  language=Bash,
  basicstyle=\ttfamily,
  commentstyle=\color{blue},
  stringstyle=\color{black},
}

\begin{document}

\akteachheader{High-Performance Scientific Computing (MATH-GA 2011/ CSCI-GA 2945)}%
{Homework Set 2}
\akteachsubheader{Due: September 19, 2012 $\cdot$ Out: September 12, 2012 }

Just like last week, turn in your homework as a repository named like
``\texttt{hpc12-hw2-netid123}'' on \url{http://forge.tiker.net}.

Also, for this homework, please ensure that your virtual machine sees
at least two (real) cores. If your personal machine is single-core,
please get in touch with us about access to a parallel machine. If you
have a CIMS account, try running on one of the machines on
\weblink{http://cims.nyu.edu/webapps/content/systems/resources/computeservers}{this
page}.

\bigskip
\akteachprobhead{Problem 1: OpenMP Deadlock}

Do problem 6.12 in Rauber/Rünger. Turn in your solution in text
form as \texttt{problem-1.txt} in your repository.

\bigskip

\akteachprobhead{Problem 2: OpenMP Bugs}

In the repository at \url{https://github.com/hpc12/hw2-problem2},
you'll find six OpenMP programs, each of which has one or more bugs,
detailed in the comments at the start of the source file.

Note that these are well-known programs, and it's super-simple to find
solutions for them on the web. We're aware of that. To state the
obvious, you'll get more out of this problem if you try to do them on
your own.

\emph{Hint for program 4:} How much stack space is needed for the
array? What does the command `\texttt{limit}' tell you about the
limit on the stack size? What does that command do? (No idea what I've just said?
Check \weblink{https://en.wikipedia.org/wiki/Stack-based_memory_allocation}{Wikipedia}.
This problem is
\weblink{http://linuxtoosx.blogspot.ch/2010/10/stack-overflow-increasing-stack-limit.html}{more
complicated} on OS X.)

Turn in a fixed version of each program with the same file name as in
the source repository above, all in a subdirectory `\texttt{problem-2}'
of your repository. Try to make sure that the output of
\begin{lstlisting}
diff -u original/omp_bugN.c hpc12-hw2-netid123/problem-2/omp_bugN.c
\end{lstlisting}
makes sense, because that's what I'll be looking at.
(Check \weblink{https://en.wikipedia.org/wiki/Diff}{this article} to
learn how to read the output.)
In particular, make as few changes as possible, and don't change
the formatting of the program.

\bigskip
\akteachprobhead{Problem 3: Build a tree in parallel}

In this problem, we will parallelize the sorted binary tree build from
homework 1.

\begin{enumerate}[a)]
  \item Add timing around the tree building part of your code. (Don't
  time the output or the generation of the random numbers.)
  Use the timing code from
  \weblink{https://github.com/hpc12/lec1-demo}{this repository}
  to output the insertion rate as millions of insertions per second.

  \item Parallelize the tree insertion, initially without regard for
  the necessary synchronization. Verify the output of your program.

  Answer the following questions:
  \begin{enumerate}[1)]
    \item What types of failures do you observe?
    \item Explain the failures you are seeing. Can you draw any
      conclusions about how your machine performs pointer updates?
    \item What speedup do you observe on this \emph{incorrect} code?
    \item What efficiency?
    \item Are there any synchronizations hidden in your code? Perhaps
      in any of the C library functions you are using? Which ones?
  \end{enumerate}
  (Use timing data for 10 million entries. If you know how caches
  work, you may check what happens on smaller/larger cases for your
  own entertainment.)

  \item Now write a version of your code that uses a critical section
  for synchronization. Verify that the output of your program is
  correct.
  \begin{enumerate}[1)]
    \item What speedup do you observe on your \emph{correct} code?
    \item What efficiency?
    \item Explain the performance, relative to the previous
    parts.
  \end{enumerate}
  (Again use timing data for 10 million entries.)

  \item Now write a version of your code that uses per-entry locks
  for synchronization. Verify that the output of your program is
  correct.
  \begin{enumerate}[1)]
    \item What speedup do you observe on your \emph{correct} code?
    \item Explain the performance, relative to the previous
    parts.
  \end{enumerate}
  (Again use timing data for 10 million entries.)

  \item Now write a version of your code that uses atomic operations
  for synchronization. Verify that the output of your program is
  correct.
  \begin{enumerate}[1)]
    \item What speedup do you observe on your \emph{correct} code?
  \end{enumerate}
  (Again use timing data for 10 million entries.)
\end{enumerate}

In a subdirectory `\texttt{problem-3}', turn in the following:
\begin{itemize}
  \item An updated \texttt{tree-sort-wrong.c} with no synchronization.
  \item A version of your code called \texttt{tree-sort-critical.c} with critical
  sections.
  \item A version of your code called \texttt{tree-sort-locks.c} with locks.
  \item A version of your code called \texttt{tree-sort-atomic.c} with atomic operations.
  \item A plain text file \texttt{answers.txt} where you provide
  answers to the prompts in parts b), c), d), e).
\end{itemize}

\end{document}

