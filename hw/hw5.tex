\documentclass[11pt]{article}

\usepackage[letterpaper,margin=1in]{geometry}

\usepackage{akteach}
\usepackage{amsmath}
\usepackage{amssymb}
\usepackage{enumerate}
\usepackage{keystroke}

\usepackage{natbib}
\newcommand{\doi}[1]{\href{http://dx.doi.org/#1}{doi: #1}}

\lstset{
  language=Bash,
  basicstyle=\ttfamily,
  commentstyle=\color{blue},
  stringstyle=\color{black},
}

\begin{document}

\akteachheader{High-Performance Scientific Computing (MATH-GA 2011/ CSCI-GA 2945)}%
{Homework Set 5}
\akteachsubheader{Out: October 10, 2012 $\cdot$ Due: October 17, 2012}


\bigskip
\akteachprobhead{Problem 1: Pitch a project}

Please form an initial idea of what final project you would like to
do. The scope should at least amount to about four homework sets in
terms of overall work, and you'll need to be able to present your
progress in a 15-minute talk towards the end of the class. The project
has to involve some non-trivial form of parallel programming. Note
that you don't need to have the idea fully developed right now. This
pitch is about me helping you to figure out whether your proposed
project is suitable. At this stage, ``no'' is an acceptable answer to
that question, and if that happens, we'll figure out a plan together.

If you'd like to work in a team, please also find your teammates now,
and pitch a project together.

\begin{enumerate}[a)]
  \item Please make an appointment with me for the week of
    October 15--October 19. During this appointment (which will last
    for about 10 minutes), I'll ask you to present, within three to
    five minutes, what your project is about, and what makes it
    interesting as a computational problem. You should also be
    prepared to answer the following questions:
    \begin{itemize}
      \item What type of parallelization do you intend to apply? Why
        do you think that type would be a good fit?
      \item What software exists for this problem? What existing
        components do you intend to use?
    \end{itemize}
    I'll provide feedback, and we'll be on our way to narrowing down
    what your project will be.
\end{enumerate}

\bigskip
\akteachprobhead{Problem 2: Simulate fish in a long tank}

In this problem, you'll be using MPI to model a large school of fish that are
swimming around a two-dimensional tank. If $N$ is the number of ranks
in your simulation, then the dimension of the tank is $50N \times 50$.
Each rank simulates fish in a portion of the tank. Rank $i$ manages the
portion $[50i,50(i+1))\times [0,50)$, i.e. from $x$-coordinate $50i$ up to
(but not including) $x$-coordinate $50(i+1)$.

\begin{enumerate}[a)]
  \item Create a structure representing a fish. A fish has
    \begin{itemize}
      \item A position $(x,y)$, represented as two double-precision
        numbers.
      \item A velocity $(vx,vy)$, represented as two double-precision
        numbers.
    \end{itemize}
    The simulation proceeds in time steps. A fish moves from $(x,y)$
    to $(x+vx, y+vy)$ every time step.

    The upper and lower walls of the tank are solid. If a fish would
    hit or penetrate a wall in a time step, it stays where it is and
    changes the sign of its vertical direction instead.

    Examine the first command line argument (call it $m$) and have
    each rank place $m$ fish randomly and uniformly in its part of the
    tank. Pick the each of the velocities uniformly in the interval
    $[-2,2]$.

  \item Write a time stepping procedure. Note that the tank is
    periodic in the horizontal direction. I.e. if a fish swims out of
    the left boundary of the tank area managed by rank $0$, it
    reappears on the right-hand side of the tank of rank $N-1$.

    Use MPI to realize the inter-rank communication necessary to
    ensure that fish keep swimming as if nothing had happened if they
    encounter the boundary between areas of the tank that are managed
    by different ranks. The presence of the computational boundaries
    should not affect the simulation at all.

    You may want to use non-blocking communication to avoid deadlocks
    and communication ordering issues. Your code has to work for any
    number of ranks.

    You'll also have to deal with the fact that a variable number of
    fish may wander from one sub-tank to the next in a time step.
    You may use fixed-size buffers for this. If you do, your code
    \emph{must} notice if it's overrunning a buffer, drop the fish
    that don't fit into the send buffer from the simulation, and print
    a comment to the console about this fact.

    If you like, you may investigate the use one of two possible
    solutions to this issue:
    \begin{itemize}
      \item \texttt{MPI\_Probe} can get the size of a message without
        actually receiving it.
      \item You can send a message containing the number of fish to be
        sent \emph{before} actually sending the fish data.
    \end{itemize}
    To send the fish, you may either send a `bag of bytes' or use
    MPI's facilities for structured data types.

  \item Add code so that each rank outputs the current state of its
    fish to a 50x50 pixel PPM image file with a name as output by
    \begin{lstlisting}
      sprintf(ppm_file_name, "fish-step%04d-rank-%04d.ppm", step, rank);
    \end{lstlisting}
    Fill the image with a black background and draw a white pixel for
    every fish, where the extent of your rank's part of the tank is
    mapped to the 50x50 pixel image.

    You may then use the following command to glue these images
    together horizontally:
    \begin{lstlisting}
      for step in $( seq -f "%04g" 0 100) ; do
        montage fish-step$step-rank*.ppm -mode Concatenate fish-step$step.png
      done
    \end{lstlisting}%stopzone
    You may have to change the number 100 to the actual number of time
    steps.

    Then delete all the PPMs and view the \texttt{.png} files using
    the \texttt{ristretto} image viewer on your VM. This allows you to
    step forward through the images in the style of a flip book.

    Use the second command line argument to determine when to output
    these image files. If that argument (call it $n$) is zero, do not
    output anything.  If it is nonzero, output the images every $n$
    steps.

    Activate this drawing code if the second command line argument is
    a \texttt{1}.

  \item Use the third command line argument to determine how many time
    steps to simulate. Insert barriers around your code, and have 
    rank 0 output a simulation timing at the end, in units of `fish
    updates per second', where this counts the total number of fish.
\end{enumerate}
Turn in your code as \texttt{fish.c} along with a \texttt{Makefile}.

\end{document}
