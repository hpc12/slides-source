\documentclass[11pt]{article}

\usepackage[letterpaper,margin=1in]{geometry}

\usepackage{akteach}
\usepackage{amsmath}
\usepackage{amssymb}
\usepackage{enumerate}
\usepackage{keystroke}

\usepackage{natbib}
\newcommand{\doi}[1]{\href{http://dx.doi.org/#1}{doi: #1}}

\lstset{
  language=Bash,
  basicstyle=\ttfamily,
  commentstyle=\color{blue},
  stringstyle=\color{black},
}

\begin{document}

\akteachheader{High-Performance Scientific Computing (MATH-GA 2011/ CSCI-GA 2945)}%
{Homework Set 4}
\akteachsubheader{Out: October 4, 2012 $\cdot$ Due: October 10, 2012}

\bigskip
\akteachprobhead{Problem 1: Blur an image with OpenCL}

In this problem, we're going to make an image ``blurry'' by replacing
each pixel's value with a weighted average of its neighboring pixels.
The weighting of the neighbors is given by this picture (which we'll
call the
\weblink{https://en.wikipedia.org/wiki/Integral_kernel}{\emph{kernel}},
not to be confused with the OpenCL kernel). You can find it in the
\weblink{https://github.com/hpc12/hw3-ppm}{PPM repository} as
\texttt{gaussian-kernel.ppm}:
\begin{center}
  \includegraphics[width=6cm]{gaussian-kernel.png}

  \textbf{Figure:} The blurring ``kernel'', an
  image which will provide the weights for our weighted
  average.
\end{center}
This image was made by \texttt{make-gaussian-kernel.c} in the same
repository, which you can feel free to play with.

If you're on the hunt for fancy words, this whole process of `compute
weighted average of neighbors for each pixel' is called a
`\weblink{https://en.wikipedia.org/wiki/Convolution}{convolution}'.
We let
\begin{itemize}
  \item $f(i,j)$ be the pixel value of the unblurry image at position $(i,j)$ with $(0,0)$
    being the upper left,
  \item $W$ and $H$ be the width and height of the unblurry image,
  \item $k$ be the width of the kernel in pixels (which
    we'll assume is even), and
  \item $K(i,j)$ be the value of the red channel of the
    kernel image such that $K(0,0)$ represents the center.
    We'll also let $l=k/2$.
\end{itemize}
With these definitions we'll be computing the `convolved' image $g$ as
\[
  g(i,j)=\frac{1}{\bar K} \sum_{m,n=-l}^{l-1} K(m,n)f(i-m,j-n)
  \qquad
  (l\le i <W-l,l\le j<H-l)
\]
where
\[
  \bar K:=\sum_{m,n=-l}^{l-1} K(m,n).
\]
\begin{enumerate}[a)]
  \item Load \texttt{gaussian-kernel.ppm} into memory. We will only be
    using information from the red channel.

    Precompute $\bar K$ for the kernel you load in (on the host). For
    \texttt{gaussian-kernel.ppm} it should come out to 48424.

  \item Read a file name from the command line and read the image
    given by that file name. For the rest of this problem, you may
    assume that both width and height are divisible by 16. If you do
    assume this, make sure you check for it and print an error if it's
    not satisfied.

    If you need an image to test with, just use one that you like (a
    family photo, say, or one of the Mandelbrot images from last time)
    and convert it to PPM by writing
    \begin{lstlisting}
    convert my-treasured-family-memory.jpeg test-image.ppm
    \end{lstlisting}
    \texttt{convert} is preinstalled in the virtual machine. You may
    also pass the option \texttt{-geometry WxH!} to \texttt{convert}
    if you need to change the size of the image. (The exclamation
    mark is important.)

  \item Write an OpenCL kernel that directly implements the
    convolution formula above. Here and in the following parts, make
    sure to implement the formula using \texttt{floats} by converting
    all values to that type.  Convert back to \texttt{unsigned char}
    (the image channel data type) on output. Do this for each channel
    of your image, and write the resulting image to
    \texttt{blurry.ppm}.

    Do not yet worry about the boundaries. In other words, only consider
    pixels of your image that are at least $l$ pixels away from the
    boundary.

    Use workgroups of size $16\times 16$.
  \item Write an OpenCL kernel that loads the kernel $K$ into local
    memory before it starts work. Make sure to use synchronization as
    needed. Again, carry out the convolution for each channel
    and write the resulting image to \texttt{blurry-local.ppm}.

    Use workgroups of size $16\times 16$.

  \item Change the kernel so that it does the right thing at the
    boundaries, too. Specifically, the average should not take into
    account pixels ``outside'' the picture. Note that you must also
    adjust $\bar K$ for these corner cases. Again, carry out the 
    convolution for each channel
    and write the resulting image to \texttt{blurry-local-boundary.ppm}.

    Use workgroups of size $16\times 16$.

  \item Time the execution of all of the above kernels. Make sure you
    wait for the compute device at all the right spots. Print
    performance data in terms of pixels/s for all kernels.

  \item Once again make sure you free/release all your buffers,
    command queues, host memory, and whatever other resources you've
    used.

    Also make sure that you check for error returns on all functions
    that can fail, including \texttt{malloc}, the \texttt{OpenCL}
    interface functions and the image reading/allocation functions.
\end{enumerate}

Turn in a main C file \texttt{convolution.c} along with
kernel files \texttt{convolution.cl},
\texttt{convolution-local.cl},
\texttt{convolution-local-boundaries.cl}. Make sure that
\texttt{convolution.c} exercises all parts of this assignment when
compiled and run.

Also, please make sure to \emph{not} check any large image files
into git. (You'll kill my server if you do. I'm not kidding.)

\end{document}
