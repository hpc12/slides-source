\documentclass[11pt]{article}

\usepackage[letterpaper,margin=1in]{geometry}

\usepackage{akteach}
\usepackage{amsmath}
\usepackage{amssymb}
\usepackage{enumerate}
\usepackage{keystroke}

\usepackage{natbib}
\newcommand{\doi}[1]{\href{http://dx.doi.org/#1}{doi: #1}}

\lstset{
  language=Bash,
  basicstyle=\ttfamily,
  commentstyle=\color{blue},
  stringstyle=\color{black},
}

\begin{document}

\akteachheader{High-Performance Scientific Computing (MATH-GA 2011/ CSCI-GA 2945)}%
{Homework Set 3}
\akteachsubheader{Out: September 26, 2012 $\cdot$ Due: October 3, 2012 }

This homework set lets you practice simple uses of OpenCL. There is an
OpenCL implementation available on your virtual machine, but don't
expect it to be blazingly fast.  Since running on GPUs is a large part
of the point of OpenCL, at the end of this assignment you can find
some instructions for running your code on GPUs available on NYU's HPC
clusters. For now, this is an optional (but highly recommended) part
of this assignment. Since we'll be using these machines more over
time, it'll be helpful to get started on using them now. In order to use
them, you'll need an HPC account. To get one, you may need to request
one at the \weblink{https://wikis.nyu.edu/display/NYUHPC/Request+or+Renew}{account request page}.

You may use the \weblink{https://github.com/hpc12/lec4-demo}{Lecture 4
demo code} as a starting point. (Make sure to look at the files ending
in \texttt{-soln.c}, not the intentionally problematic `initial'
codes.

\bigskip
\akteachprobhead{Problem 1: Image + OpenCL warm-up}

Read in two numbers $w$ and $h$ from the command line and allocate three
buffers \texttt{red}, \texttt{green}, \texttt{blue} on the compute
device, each sufficient for $w\times h$ \texttt{unsigned char}s.


Write an OpenCL kernel that fills the buffers above with
linearly blended colors such that
\begin{itemize}
  \item The red channel is zero on the left-hand side of your
    image and 255 (the maximum channel value) on the right-hand side
    of your image.
  \item The blue channel is zero on the top of your
    image and 255 (the maximum channel value) on the bottom
    of your image.
  \item The green channel is zero everywhere.
\end{itemize}
Write your kernel with a work group size of $1\times 1$ and a
global size of $w\times h$.

Use the library routines provided in
\weblink{https://github.com/hpc12/hw3-ppm}{this repository} to
write a \weblink{https://en.wikipedia.org/wiki/Netpbm_format}{PPM}
(``portable pixmap'') image file called \texttt{linear-blend.ppm}.
The file \texttt{test-ppm.c} demonstrates the use of the library.

You may look at the resulting image by typing
\begin{lstlisting}
display linear-blend.ppm
\end{lstlisting}

This command is preinstalled on your virtual machine. If you're not
using the VM, find an image viewer that can display PPM images.

Note that images are conventionally represented in row-major
order. In other words, the $x$ coordinate varies fastest in
memory.

Turn in a main C file \texttt{problem-1/blend.c} along with
a kernel file \texttt{problem-1/linear-blend.cl}. Also turn in
a Makefile that builds your code.

Also, please make sure to \emph{not} check the generated image files
into git. (You'll kill my server if you do. I'm not kidding.)

\bigskip
\akteachprobhead{Problem 2: Compute a Mandelbrot set}

The \weblink{https://en.wikipedia.org/wiki/Mandelbrot_set}{Mandelbrot
set} is a subset of the complex numbers that has a surprisingly simple
mathematical definition, but a surprisingly complicated mathematical
structure. (Don't know complex numbers?  Don't panic, see below.)

Somewhat formally, it is defined as
\[
  \{c \in \mathbb C: \text{The sequence defined by $z_0=0$,
  $z_n=z_{n-1}^2 +c$ becomes $> 2$ in magnitude for some $n$}\}.
\]
If your complex number skills are a bit rusty: A complex number
consists of two real numbers (think \texttt{float}s) $z=(a,b)$.
If we let $c=(x,y)$ and $z_n=(a_n,b_n)$ (for $n\ge 0$), then the
above formula can be expressed as
\begin{align*}
  a_n &= x-b_{n-1}^2+a_{n-1}^2\\
  b_n &= y+2a_{n-1}b_{n-1}
\end{align*}

\begin{enumerate}[a)]
  \item Read in two numbers $w$ and $h$ from the command line and allocate three
    buffers \texttt{red}, \texttt{green}, \texttt{blue} on the compute
    device, each sufficient for $w\times h$ \texttt{unsigned char}s.

  \item Write an OpenCL kernel that carries out the Mandelbrot
    iteration for $c=(x,y)$. Let $x$ vary linearly from
    \texttt{xleft} and \texttt{xright} on the left and right, and
    let $y$ vary linearly from \texttt{ytop} and \texttt{ybottom},
    on the respective ends of your image. Pass these values as
    parameters to your kernel.

    Use the following values for the work you turn in:
    \begin{center}
      \begin{tabular}{|l|l|}
        \hline
        \texttt{xleft} & $-2.13$ \\
        \hline
        \texttt{xright} & 0.77 \\
        \hline
        \texttt{ytop} & 1.3 \\
        \hline
        \texttt{ybottom} & $-1.3$\\
        \hline
      \end{tabular}
    \end{center}
    But feel free to play around.

    For each pixel of the image, run at most \texttt{max\_iter}
    iterations, where that is another parameter to your kernel. Stop
    the iteration when the square of the magnitude of the iterate
    reaches greater than four. For a complex number $z=(a,b)$, the
    square of the magnitude is computed as $|z|^2=a^2+b^2$.

    Store the number of iterations until four is exceeded in all
    channels (red, green, blue) of your image, in such a way that if
    you used \texttt{max\_iter} iterations, the value becomes 255.
    (You may use more `interesting' color maps, too, if you like.)
    Store the resulting image as \texttt{mandelbrot.ppm}.

    Write your kernel with a work group size of $1\times 1$ and a
    global size of $w\times h$.
    \label{part:mbrot-small-wg}

  \item Modify your OpenCL kernel so that it uses a workgroup size of
    $16\times 16$, but make sure that your program can still be run
    with any $w$ and $h$. (I.e. you will have to handle edge cases.)
    Store the resulting image as \texttt{mandelbrot-large-wg.ppm}.
    \label{part:mbrot-large-wg}

  \item Use the \texttt{timing.h} infrastructure from the last two
    homeworks to time the execution of both the small-workgroup and
    the large-workgroup case. Make sure to wait for the compute device
    to finish its job in the right places. Output the performance in
    millions of pixels per second for each case, to three decimal
    places.

  \item Make sure you free/release all your buffers, command queues, host
    memory, and whatever other resources you've used.

    Also make sure that you check for error returns on all functions
    that can fail, including \texttt{malloc} and \texttt{OpenCL}
    interface functions.
\end{enumerate}

Turn in a main C file \texttt{problem-2/mandelbrot.c} along with kernel files
\texttt{problem-2/small-wg.cl} and \texttt{problem-2/large-wg.cl},
corresponding to parts \ref{part:mbrot-small-wg}) and
\ref{part:mbrot-large-wg}). Make sure that \texttt{mandelbrot.c} exercises all
parts of this assignment as described above when compiled and run.  Also turn
in a Makefile that builds your code.

Lastly, please make sure to \emph{not} check the generated image files
into git. (You'll kill my server if you do. I'm not kidding.)

\clearpage
% -----------------------------------------------------------------------------
\akteachheader{High-Performance Scientific Computing (MATH-GA 2011/ CSCI-GA 2945)}{Access to the NYU clusters}
% -----------------------------------------------------------------------------
\subsection*{Logging into a cluster}

All access to the high-performance machines at NYU goes through
\texttt{hpc.nyu.edu} (a.k.a. the ``bastion host''). You may log in using
\weblink{http://en.wikipedia.org/wiki/Secure_Shell}{SSH}:
\begin{lstlisting}
your-machine$ ssh NETID@hpc.nyu.edu
\end{lstlisting}%stopzone

Your user name on this machine is your NYU NetID (the same one you use
to log into NYUHome). Likewise, your password is the same one you use
to access NYUHome.

\begin{note}
As before, the dollar sign ``\texttt{\$}'' represents a
command prompt. The actual command you need to type follows after that. For
clarity, a symbolic host name may precede the dollar sign.
\end{note}

The virtual machine comes with \texttt{ssh} pre-installed.  If you are
using a Windows machine, you may use
\weblink{http://www.chiark.greenend.org.uk/~sgtatham/putty/}{PuTTY}.

The first time you log in, the system will warn you that it does not
know about \texttt{hpc.nyu.edu}; just say ``yes'' to log in and
permanently accept this machine as a known host.

From the bastion host, more or less the only thing you can do is use
SSH to access other machines. To reach  the ``cuda'' cluster, type
\begin{lstlisting}
hpc$ ssh cuda
\end{lstlisting}%stopzone
To reach the ``Bowery''  cluster, type
\begin{lstlisting}
hpc$ ssh bowery
\end{lstlisting}%stopzone
The first time you log in to each, you will again get a warning from
\texttt{ssh}. Proceed as above.

The first time you log in, you will also be prompted to set up an SSH key
pair.  This key pair serves two purposes: First, it will
allow your jobs to access the cluster's compute nodes on which they
are supposed to run. Second, it will act as your key to the class
collaboration space when logged in to the NYU clusters.

Just hit \Enter when prompted for a passphrase. The interaction will
look something like this:

\begin{lstlisting}
It doesn't appear that you have set up your ssh key.
This process will make the files:
     /home/NETID/.ssh/id_rsa.pub
     /home/NETID/.ssh/id_rsa

Generating public/private rsa key pair.
Enter file in which to save the key (/home/NETID/.ssh/id_rsa):
Enter passphrase (empty for no passphrase):
Enter same passphrase again:
Your identification has been saved in /home/NETID/.ssh/id_rsa.
Your public key has been saved in /home/NETID/.ssh/id_rsa.pub.
The key fingerprint is: ...
\end{lstlisting}

Now you are logged into the cluster!  More precisely, you will be
logged into the cluster's so-called `head node'. You may compile your
programs there, but you are not supposed to run large computations on
that machine. See the instructions below on how to start computation
jobs.

% -----------------------------------------------------------------------------
\subsection*{Editing files on the clusters}

A simple, friendly, somewhat bare-bones editor is available by typing
\begin{lstlisting}
$ nano filename.c
\end{lstlisting}%stopzone
% -----------------------------------------------------------------------------
\subsection*{Moving files to and from NYU HPC}

See this
\weblink{https://wikis.nyu.edu/display/NYUHPC/SCP+through+SSH+Tunneling}{wiki
page} on how to move data back and forth between your computer and the
HPC clusters. (Note that if you're moving source code, git via forge is usually
more convenient.)

Note that if you have access to a machine that has SCP/SFTP accessible
from the outside network (such as \texttt{access.cims.nyu.edu}), you
can avoid the tedious two-step process via the bastion host by
initiating file transfers \emph{from} the clusters.

% -----------------------------------------------------------------------------
\subsection*{Using \texttt{git} on the clusters}

To use
\texttt{git} on `cuda', you need to type
\begin{lstlisting}
$ module load git/gnu/1.7.2.3
\end{lstlisting}%stopzone
and on `bowery', you need to type
\begin{lstlisting}
$ module load git/intel/1.7.6.3
\end{lstlisting}%stopzone
To avoid having to type this every time, consider adding this command
to the \texttt{.bashrc} (note the dot!) file in your home directory.

NYU uses the software ``\weblink{http://modules.sourceforge.net/}{Environment
Modules}'' to manage a variety of software installed on its clusters.
By typing the `\texttt{module load}' command above, you have already
used this system to allow you to use \texttt{git}. You may type
\begin{lstlisting}
module avail
\end{lstlisting}
to find which other software modules are available. In particular, you
may choose to use the
\weblink{http://en.wikipedia.org/wiki/Intel\_C\%2B\%2B\_Compiler}{Intel
Compilers} instead of the default GNU ones. Like \texttt{git}, these
compilers are activated through the \texttt{module load} commands. The
NYU HPC Wiki describes these compilers in more detail.

Next, inform \texttt{git} of your name and email address, as before:
\begin{lstlisting}
$ git config --global user.name "Your Name Here"
$ git config --global user.email you@yourdomain.example.com
\end{lstlisting}

Next, while you are logged into each clster, type
\begin{lstlisting}
cluster$ cat $HOME/.ssh/id_rsa.pub
\end{lstlisting}
This will print a few lines (technically, just one wrapped line) that
looks like this:
\begin{lstlisting}
ssh-rsa AAAABw...vM3XdIbZWwmXH/iNbFWEhZw== NETID@login-0-1.local
\end{lstlisting}
Once you are logged into \texttt{forge}, click your name in the top
left corner. Click ``Update your account'' on the left and then paste
the line returned above into the field that says ``Add public key''.
Click ``Update your Account''. The key update may take up to two
minutes to fully complete, so if any steps involving git below fail,
wait for two minutes, and then try again.
This completes the setup steps that are only required once for each
cluster.

Next, you may `clone' an existing homework repository by using
\begin{lstlisting}
git clone ssh://git@forge.tiker.net:2234/hpc12-hw3-netid123.git
\end{lstlisting}
You may make some changes here, do \verb|git add| and \verb|git commit|.
\verb|git push| will upload them `to the cloud'. Likewise,
\verb|git pull| will download. That should help you get started. Git
will also be next week's `tool of the week' in the lecture.

% -----------------------------------------------------------------------------
\subsection*{Running code on the clusters}
Here is a sample transcript that starts just after I logged in to the `cuda'
cluster:

\begin{lstlisting}
# First, we're getting a list of all `queues' supplied by the scheduler
# on `cuda'.
[ak177@cuda ~]$ qstat -Q
Queue              Max   Tot   Ena   Str   Que   Run   Hld   Wat   Trn   Ext T
----------------   ---   ---   ---   ---   ---   ---   ---   ---   ---   --- -
serial               0     0   yes   yes     0     0     0     0     0     0 E
interactive          0     4   yes   yes     0     0     0     0     0     0 E
default              0     0   yes   yes     0     0     0     0     0     0 E
batch                0     0   yes   yes     0     0     0     0     0     0 E
p12                  0     0   yes   yes     0     0     0     0     0     0 E

# Next, we're submitting an interactive job (-I) to a queue
# named `interactive' (-q).
[ak177@cuda ~]$ qsub -q interactive -I
qsub: waiting for job 2113.cuda.es.its.nyu.edu to start
qsub: job 2113.cuda.es.its.nyu.edu ready

# Even if you loaded the git module on the head node, you may have to do this
# again now, because you're now logged into a new machine, a `compute node'.
# Observe how the prompt has changed.
[ak177@compute-0-3 ~]$ module load git/gnu/1.7.2.3

# We'll grab the lecture demos...
[ak177@compute-0-3 ~]$ git clone git://github.com/hpc12/lec4-demo
Cloning into lec4-demo...
remote: Counting objects: 65, done.
remote: Compressing objects: 100% (39/39), done.
remote: Total 65 (delta 34), reused 56 (delta 25)
Receiving objects: 100% (65/65), 30.94 KiB, done.
Resolving deltas: 100% (34/34), done.
[ak177@compute-0-3 ~]$ cd lec4-demo/

# ...and build them.
[ak177@compute-0-3 lec4-demo]$ make
gcc -c -std=gnu99 cl-helper.c
gcc -std=gnu99 -lrt -lOpenCL -ovec-add vec-add.c cl-helper.o
gcc -std=gnu99 -lrt -lOpenCL -ovec-add-soln vec-add-soln.c cl-helper.o
gcc -std=gnu99 -lrt -lOpenCL -oprint-devices print-devices.c cl-helper.o
gcc -std=gnu99 -lrt -lOpenCL -otranspose transpose.c cl-helper.o
gcc -std=gnu99 -lrt -lOpenCL -otranspose-soln transpose-soln.c cl-helper.o

# Let's see what devices we have here:
[ak177@compute-0-3 lec4-demo]$ ./print-devices
platform 0: vendor 'NVIDIA Corporation'
  device 0: 'GeForce GTX 285'
platform 1: vendor 'Advanced Micro Devices, Inc.'
  device 0: 'Intel(R) Xeon(R) CPU           E5405  @ 2.00GHz'

# Let's run the vector addition demo.
[ak177@compute-0-3 lec4-demo]$ ./vec-add-soln 10000000 10
create_context_on: specified device not found.
Aborted

# Ok, the code is still looking for the "Intel" platform, which doesn't
# appear to exist here. Let's change "Intel" to "NVIDIA":
[ak177@compute-0-3 lec4-demo]$ nano vec-add-soln.c

# Rebuild the code:
[ak177@compute-0-3 lec4-demo]$ make
gcc -std=gnu99 -lrt -lOpenCL -ovec-add-soln vec-add-soln.c cl-helper.o

# And off we go:
[ak177@compute-0-3 lec4-demo]$ ./vec-add-soln 10000000 10
0.001777 s
135.051488 GB/s
GOOD

# This command returns you to the head node and frees up the machine
# for other people. Since the machines are a shared resource, it's
# common courtesy to log out if you're not actively using the compute node
# you're logged # into. (It's fine to stay logged into the head node.)
[ak177@compute-0-3 lec4-demo]$ exit
\end{lstlisting}

\end{document}
