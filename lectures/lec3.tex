\documentclass[english,compress]{beamer}
\usepackage{kloeckislides}
\nonstopmode

\usepackage[normalem]{ulem}
%\useoutertheme{split}
%\useinnertheme{rectangles}
%\usecolortheme{owl}
%\usetheme{Singapore}

\usepackage{pifont}
\usepackage{ifthen}

\setbeamercolor{section in head/foot}{use=structure,bg=structure.fg!25!bg}
\defbeamertemplate*{footline}{split theme}
{%
  \leavevmode%
  \begin{beamercolorbox}[wd=.5\paperwidth,ht=2.5ex,dp=1.125ex]{section in head/foot}%
    \insertsectionnavigationhorizontal{\paperwidth}{\hskip0pt plus1filll}{}%
  \end{beamercolorbox}%
  %\begin{beamercolorbox}[wd=.5\paperwidth,ht=2.5ex,dp=1.125ex]{subsection in head/foot}%
    %\insertsubsectionnavigationhorizontal{.5\paperwidth}{}{\hskip0pt plus1filll}%
  %\end{beamercolorbox}%
}


%\useoutertheme[subsection=false]{miniframes}

\setbeamertemplate{frametitle}[default][center]

\AtBeginDocument{%
  {
    \usebeamercolor{section in head/foot}
  }
  
  \pgfdeclareverticalshading{beamer@headfade}{\paperwidth}
  {%
    color(0cm)=(bg);
    color(1.25cm)=(section in head/foot.bg)%
  }

  \setbeamercolor{section in head/foot}{bg=}
}

\addtoheadtemplate{\pgfuseshading{beamer@headfade}\vskip-1.25cm}{}

\beamertemplatedotitem

\setbeamercolor{section in head/foot}{parent=palette quaternary}
\setbeamercolor{subsection in head/foot}{parent=palette primary}

\setbeamercolor{author in head/foot}{parent=section in head/foot}
\setbeamercolor{title in head/foot}{parent=subsection in head/foot}



\AtBeginSection[] {
  \begin{frame}<beamer>
  \frametitle{Outline}
  \tableofcontents[sectionstyle=show/shaded,subsectionstyle=show/show/hide]
\end{frame}
}
\AtBeginSubsection[] {
  \begin{frame}<beamer>
  \frametitle{Outline}
  \tableofcontents[sectionstyle=show/shaded,subsectionstyle=show/shaded/hide]
\end{frame}
}

\newcommand{\technicality}[2]{%
  {\strut #1\\
    \begin{beamercolorbox}[sep=1mm]{block body}
      #2
    \end{beamercolorbox}
  }%
}

\lstset{
  language=C++,
  rangebeginprefix=//\ ,
  rangeendprefix=//\ ,
}

\def\weblink#1#2{\href{#1}{\color{blue}\underline{#2}}}

\definecolor{fetch}{RGB}{227,110,35}
\definecolor{alu}{RGB}{255,188,24}
\definecolor{context}{RGB}{132,146,175}

\usepackage{keystroke}

\setbeamertemplate{navigation symbols}{}


\begin{document}
% {{{ front matter

\title{High-Performance Scientific Computing\\Lecture 3: OpenCL}

\date{MATH-GA 2011 / CSCI-GA 2945 $\cdot$ September 19, 2012}

\frame{\titlepage}

\begin{frame}{Today}
  \tableofcontents[hideallsubsections]
\end{frame}
% }}}
% -----------------------------------------------------------------------------
\begin{frame}{Admin Bits}
  \begin{itemize}[<+->]
    \item New here? Please send email
    \item Started looking for a final project yet?
    \item HW1 not found $\rightarrow$ email
    \item Grading
    \item Overall pace
    \item HW3 out on the weekend
  \end{itemize}
\end{frame}
% -----------------------------------------------------------------------------
\section{HW2}
% -----------------------------------------------------------------------------
\begin{frame}{HW2 problem 2}
  \begin{center}
  \Huge Demo time
  \end{center}
\end{frame}
% -----------------------------------------------------------------------------
\begin{frame}{OpenMP sync primitives}
  \begin{itemize}[<+->]
    \item Critical section
    \item Locks
    \item Atomics
      \begin{itemize}
        \item Update: \texttt{x++;}
        \item Capture: \texttt{v = x++;}
        \item Structured: {v = x; x |= expr;} (``Test-and-set'')
        \item Compare-and-swap (not in OpenMP)
      \end{itemize}
  \end{itemize}
\end{frame}
% -----------------------------------------------------------------------------
\begin{frame}{OpenMP corner case pop quiz 1}
  \begin{itemize}[<+->]
    \item May OpenMP directives be nested?
      \begin{itemize}
        \item What is an orphaned directive?
        \item What is close nesting?
        \item What is a `dynamic extent' of a region?
      \end{itemize}
    \item May a worksharing region be closely nested inside
      another one?
    \item What happens if I nest two \texttt{critical} regions
      of the same name?
  \end{itemize}
\end{frame}
% -----------------------------------------------------------------------------
\begin{frame}{OpenMP corner case pop quiz 2}
  \begin{itemize}[<+->]
    \item Corresponding getter function for
      \texttt{omp\_set\_num\_threads()}?
    \item Relation between
      \texttt{omp\_set\_dynamic()} and \texttt{schedule(dynamic)}?
    \item What is wrong with this statement?
      \begin{quote}
        A barrier region may not be closely nested inside a
        worksharing region. \hfill {\footnotesize (from the OpenMP tutorial)}
      \end{quote}
    \item What threads does a \texttt{barrier} bind to?
    \item What threads does a \texttt{critical} region bind to?
  \end{itemize}
\end{frame}
% -----------------------------------------------------------------------------
\section{Chips for Throughput}
% -----------------------------------------------------------------------------
% {{{
\input{cpu-chip-real-estate}

\newcommand{\kayvoncredit}{
  \begin{tikzpicture}[overlay]
    \node [xshift=1cm,yshift=0.5cm]
      at (current page.south west)
      [font=\scriptsize,fill=gray!30,anchor=south west,opacity=0.5]
      {Credit: Kayvon Fatahalian (Stanford) };
  \end{tikzpicture}
}
\newcommand{\kayvonframe}[5]{

  \begin{frame}{#1}
    #4
    \begin{center}
    \includegraphics[viewport=#3,clip=true,page=#2,height=0.7\textheight]{kayvon-gpuarch.pdf}
    \end{center}
    \kayvoncredit
    #5
  \end{frame}
}

\kayvonframe{``CPU-style'' Cores}{13}{1in 1in 9in 6.5in }{}{}
\kayvonframe{Slimming down}{14}{1in 1in 9in 6.5in }{}{}
\kayvonframe{More Space: Double the Number of Cores}{15}{2.5in 1in 8.5in 6.5in }{}{}
\kayvonframe{\dots again}{16}{4in 1in 9.5in 6.5in }{}{}
\kayvonframe{\dots and again}{17}{4in 1.35in 9.5in 6.5in }{}{
  \uncover<2>{
    \begin{tikzpicture} [overlay]
      \node [above left=1cm of current page.south east, draw,drop shadow,fill=white,
      text width=0.6\textwidth, inner xsep=0.5cm,inner ysep=0.5cm,thick]
        {
          $\rightarrow$ 16 independent instruction streams

          \medskip
          Reality: instruction streams not actually
          very different/independent
        } ;
    \end{tikzpicture}
  }
}
\begin{frame}{Saving Yet More Space}
  \begin{columns}
    \column{0.5\textwidth}
      \only<1-2>{%
      \includegraphics[viewport=1.8in 3.8in 5.45in 6.25in,clip=true,page=19,width=\textwidth]{kayvon-gpuarch.pdf}\\[-0.5mm]
      \includegraphics[viewport=1.8in 1.35in 5.45in 3.8in,clip=true,page=19,width=\textwidth]{kayvon-gpuarch.pdf}
      }%
      \only<3>{%
      \includegraphics[viewport=1.8in 3.8in 5.45in 6.25in,clip=true,page=20,width=\textwidth]{kayvon-gpuarch.pdf}\\[-0.5mm]
      \includegraphics[viewport=1.8in 1.35in 5.45in 3.8in,clip=true,page=19,width=\textwidth]{kayvon-gpuarch.pdf}
      }%
      \only<4->{%
      \includegraphics[viewport=1.8in 3.8in 5.45in 6.25in,clip=true,page=20,width=\textwidth]{kayvon-gpuarch.pdf}\\[-0.5mm]
      \includegraphics[viewport=1.8in 1.35in 5.45in 3.8in,clip=true,page=20,width=\textwidth]{kayvon-gpuarch.pdf}
      }
    \column{0.5\textwidth}%
      \uncover<2->{%
        \textbf{Idea \#2}

        \medskip
        Amortize cost/complexity of managing an instruction
        stream across many ALUs

        \medskip
        \large \textbf{$\rightarrow$ SIMD}
      }
  \end{columns}
  \kayvoncredit
\end{frame}
\kayvonframe{Gratuitous Amounts of Parallelism!}{24}{5in 1.45in 10in 6.75in }{}{
  \uncover<2->{
    \begin{tikzpicture} [overlay]
      \node [below right=1.75cm of current page.north west, draw,drop shadow,fill=white,
      text width=0.8\textwidth, inner sep=2.5mm,thick]
        {
          Example:

          \medskip
          128 instruction streams in parallel

          16 independent groups of 8 synchronized streams
        } ;
    \end{tikzpicture}
  }
  \uncover<3>{
    \begin{tikzpicture} [overlay]
      \node [above left=1cm of current page.south east, draw,drop shadow,fill=white,
      text width=0.6\textwidth, inner xsep=0.5cm,inner ysep=0.5cm,thick]
        {
          Great if everybody in a group does the same thing.

          \medskip
          But what if not?

          \medskip
          What leads to divergent instruction streams?
        } ;
    \end{tikzpicture}
  }
}
\kayvonframe{Branches}{26}{0.85in 0.9in 10.5in 6.8in }{}{}
\kayvonframe{Branches}{27}{0.85in 0.9in 10.5in 6.8in }{}{}
\kayvonframe{Branches}{28}{0.85in 0.9in 10.5in 6.8in }{}{}
\kayvonframe{Branches}{29}{0.85in 0.9in 10.5in 6.8in }{}{}

\begin{frame}{Recent Processor Architecture}
  \begin{columns}
    \column{0.5\textwidth}
      \begin{itemize}
      \item Commodity chips
      \item ``Infinitely'' many cores
      \item ``Infinite'' vector width
      \item Must hide memory latency\\
        ($\rightarrow$ ILP, SMT)
      \end{itemize}
    \column{0.5\textwidth}
      \begin{itemize}
      \item Compute bandwidth\\
        \hfill $\gg$ Memory bandwidth
      \item Bandwidth only achievable by \emph{homogeneity}
      \end{itemize}
      \vspace*{3.5ex}
  \end{columns}
  \hspace*{-0.25\textwidth}
  \begin{tabular}{p{0.25\textwidth}p{0.25\textwidth}p{0.25\textwidth}p{0.25\textwidth}p{0.25\textwidth}}
  \includegraphics[width=0.25\textwidth]{gt200-die.jpg}
  &
  \includegraphics[width=0.25\textwidth]{fermi-die-shot.jpeg}
  &
  \centering
  \includegraphics[width=3.5cm,angle=90]{ivy-bridge-die-shot.jpeg}
  &
  \includegraphics[width=0.25\textwidth]{tahiti-die-shot.jpeg}
  &
  \includegraphics[width=0.25\textwidth]{gk110-die-shot.jpeg}
  \\
  \centering Nv~GT200\par(2008)
  &
  \centering Nv~Fermi\par(2010)
  &
  \centering Intel IVB\par(2012)
  &
  \centering AMD Tahiti \par(2012)
  &
  \centering Nv GK110\par(2012?)
  \end{tabular}
\end{frame}
% }}}
% -----------------------------------------------------------------------------
\section{Synchronization}
% -----------------------------------------------------------------------------
\input{barrier}
\input{memory-fence}
\input{cl-atomic}
% }}}
\questionframe{}
\imagecreditslide

\end{document}
% vim: foldmethod=marker
