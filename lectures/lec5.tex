\documentclass[english,compress]{beamer}
\usepackage{kloeckislides}
\nonstopmode

\usepackage[normalem]{ulem}
%\useoutertheme{split}
%\useinnertheme{rectangles}
%\usecolortheme{owl}
%\usetheme{Singapore}

\usepackage{pifont}
\usepackage{ifthen}

\setbeamercolor{section in head/foot}{use=structure,bg=structure.fg!25!bg}
\defbeamertemplate*{footline}{split theme}
{%
  \leavevmode%
  \begin{beamercolorbox}[wd=.5\paperwidth,ht=2.5ex,dp=1.125ex]{section in head/foot}%
    \insertsectionnavigationhorizontal{\paperwidth}{\hskip0pt plus1filll}{}%
  \end{beamercolorbox}%
  %\begin{beamercolorbox}[wd=.5\paperwidth,ht=2.5ex,dp=1.125ex]{subsection in head/foot}%
    %\insertsubsectionnavigationhorizontal{.5\paperwidth}{}{\hskip0pt plus1filll}%
  %\end{beamercolorbox}%
}


%\useoutertheme[subsection=false]{miniframes}

\setbeamertemplate{frametitle}[default][center]

\AtBeginDocument{%
  {
    \usebeamercolor{section in head/foot}
  }
  
  \pgfdeclareverticalshading{beamer@headfade}{\paperwidth}
  {%
    color(0cm)=(bg);
    color(1.25cm)=(section in head/foot.bg)%
  }

  \setbeamercolor{section in head/foot}{bg=}
}

\addtoheadtemplate{\pgfuseshading{beamer@headfade}\vskip-1.25cm}{}

\beamertemplatedotitem

\setbeamercolor{section in head/foot}{parent=palette quaternary}
\setbeamercolor{subsection in head/foot}{parent=palette primary}

\setbeamercolor{author in head/foot}{parent=section in head/foot}
\setbeamercolor{title in head/foot}{parent=subsection in head/foot}



\AtBeginSection[] {
  \begin{frame}<beamer>
  \frametitle{Outline}
  \tableofcontents[sectionstyle=show/shaded,subsectionstyle=show/show/hide]
\end{frame}
}
\AtBeginSubsection[] {
  \begin{frame}<beamer>
  \frametitle{Outline}
  \tableofcontents[sectionstyle=show/shaded,subsectionstyle=show/shaded/hide]
\end{frame}
}

\newcommand{\technicality}[2]{%
  {\strut #1\\
    \begin{beamercolorbox}[sep=1mm]{block body}
      #2
    \end{beamercolorbox}
  }%
}

\lstset{
  language=C++,
  rangebeginprefix=//\ ,
  rangeendprefix=//\ ,
}

\def\weblink#1#2{\href{#1}{\color{blue}\underline{#2}}}

\definecolor{fetch}{RGB}{227,110,35}
\definecolor{alu}{RGB}{255,188,24}
\definecolor{context}{RGB}{132,146,175}

\usepackage{keystroke}

\setbeamertemplate{navigation symbols}{}


\begin{document}
% {{{ front matter

\title{High-Performance Scientific Computing\\Lecture 5: More OpenCL, MPI}

\date{MATH-GA 2011 / CSCI-GA 2945 $\cdot$ October 3, 2012}

\frame{\titlepage}

\begin{frame}{Today}
  \tableofcontents[hideallsubsections]
\end{frame}
% }}}
% -----------------------------------------------------------------------------
\begin{frame}{Bits and pieces}
  \begin{itemize}
    \item HW1 grades sent
    \item HW2 graded soon
    \item HW3 due
    \item HW4 out tomorrow
    \item Cuda cluster accounts
    \item Mailing list messages
  \end{itemize}
\end{frame}
% -----------------------------------------------------------------------------
\section{Tool of the day: Git}
% -----------------------------------------------------------------------------
\begin{frame}{Git}
  \begin{center}
  \Huge Demo time
  \end{center}
\end{frame}
% -----------------------------------------------------------------------------
\section{OpenCL: Device Language}
% -----------------------------------------------------------------------------
\begin{frame}{Moar OpenCL!}
  \begin{center}
  \Huge Demo time
  \end{center}
\end{frame}
% -----------------------------------------------------------------------------
% {{{
\begin{frame}{OpenCL Device Language}
  \begin{columns}
    \column{0.7\textwidth}
      OpenCL device language is C99, with these differences:

      \medskip
      \plusball Index getters

      \plusball Memory space qualifiers

      \plusball Vector data types 

      \plusball Many generic (`overloaded') math functions

      \plusball Synchronization

      \minusball Recursion

      \minusball Fine-grained \texttt{malloc()}

      \minusball Function pointers
    \column{0.3\textwidth}
      \includegraphics[width=\textwidth]{opencl-logo.png}
  \end{columns}
\end{frame}

%\input{cuda-cl-dictionary}

\begin{frame}{Address Space Qualifiers}

  \begin{center}
  \begin{tabular}{p{5em}cccp{2.8cm}}
    \hline
    \textbf{Type} & \textbf{Per} & \textbf{``Speed''} \\
    \hline
    private*) & work item & super-fast\\
    local & group & fast \\
    global & grid & kinda slow \\
    \hline
  \end{tabular}

  \bigskip
  *) default, so optional
  \end{center}
  \uncover<2>{
    \begin{tikzpicture} [overlay]
      \node [
        above left=1cm of current page.south east, draw,drop shadow,fill=white,
        inner sep=5mm,thick, text width=0.6\textwidth]
        {
          Should really discuss ``speed'' in terms of
          latency/bandwidth.

          \bigskip
          \emph{Both} decrease with distance from the point of
          execution.
        } ;
    \end{tikzpicture}
  }
\end{frame}

% }}}
% -----------------------------------------------------------------------------
\section{OpenCL: Synchronization}
% -----------------------------------------------------------------------------
% {{{
\begin{nologo}
\begin{frame}{Concurrency and Synchronization}
  \uncover<+->{
    \begin{beamercolorbox}[sep=3mm]{block body}
      GPUs have layers of concurrency.\\
      \hfill Each layer has its synchronization primitives.
    \end{beamercolorbox}
  }
  \bigskip
  \begin{columns}[c]
    \column{0.6\textwidth}
      \uncover<+->{
        \begin{itemize}
          \item Intra-group:\\ 
            \texttt{barrier(\dots)}, \\
            \texttt{mem\_fence(\dots)}\\
            \texttt{\dots} =
            \texttt{CLK\_\{LOCAL,GLOBAL\}\_MEM\_FENCE}
          \item Inter-group:\\ Kernel launch
          \item CPU-GPU:\\ Command queues%, Events
        \end{itemize}
      }
    \column{0.35\textwidth}
      \includegraphics[width=\textwidth]{onion.jpeg}
  \end{columns}

\end{frame}
\end{nologo}
\addimgcredit{Onions: flickr.com/darwinbell \cc}

\input{barrier}
\input{memory-fence}

% -----------------------------------------------------------------------------
\begin{frame}{Synchronization between Groups}
  \begin{block}<+->{Golden Rule:}
    Results of the algorithm must be independent of the order in which
    work groups are executed.
  \end{block}

  \uncover<+->{
  \textbf{Consequences:}
  \begin{itemize}
    \item Work groups may read the same information from global memory.
    \item But: Two work groups may not validly write different things to 
      the same global memory.
    \item Kernel launch serves as
      \begin{itemize}
        \item Global barrier
        \item Global memory fence
      \end{itemize}
  \end{itemize}
  }
\end{frame}
% -----------------------------------------------------------------------------
\begin{frame}{Atomic Operations}
  \uncover<+->{
  Collaborative (inter-block) Global Memory Update:
  \begin{center}
  \begin{tikzpicture}[
    start chain=going right,
    node distance=1cm,
    op/.style={draw,thick,fill=blue!50,on chain},
    every join/.style={},
    ]
    \node [op] (read) {Read} ;
    \node [op] (inc) {Increment} ;
    \node [op] (write) {Write} ;
    \draw [thick,->] (read) -- (inc) coordinate [pos=0.5] (r-i);
    \draw [thick,->] (inc) -- (write) coordinate [pos=0.5] (i-w);
    \uncover<+->{
      \node at (r-i) [arrow box,fill=yellow,
        arrow box arrows={north:0.5cm},below,draw,thick]
        {Interruptible!} ;
    }
    \uncover<+->{
      \node at (i-w) [arrow box,fill=yellow,
        arrow box arrows={north:0.5cm},below,draw,thick]
        {Interruptible!} ;
    }
  \end{tikzpicture}
  \end{center}
  }
  \uncover<+->{
  Atomic Global Memory Update:
  \begin{center}
  \begin{tikzpicture}[
    start chain=going right,
    node distance=1cm,
    op/.style={draw,thick,fill=blue!50,on chain},
    every join/.style={},
    ]
    \node [op] (read) {Read} ;
    \node [op] (inc) {Increment} ;
    \node [op] (write) {Write} ;
    \draw [thick,->] (read) -- (inc) coordinate [pos=0.5] (r-i);
    \draw [thick,->] (inc) -- (write) coordinate [pos=0.5] (i-w);
    \begin{pgfonlayer}{background}
      \node [fit=(read) (write), fill=blue!30,draw,thick] { } ;
    \end{pgfonlayer}
    \uncover<+->{
      \node at (r-i) [arrow box,fill=green!75,
        arrow box arrows={north:0.5cm},below=0.3cm,draw,thick]
        {Protected} ;
    }
    \uncover<+->{
      \node at (i-w) [arrow box,fill=green!75,
        arrow box arrows={north:0.5cm},below=0.3cm,draw,thick]
        {Protected} ;
    }
  \end{tikzpicture}
  \end{center}
  }
  \uncover<+->{
    \textbf{How?}\\
    atomic\_\{add,inc,cmpxchg,\dots\}(int *global, int value);
  }
\end{frame}
% -----------------------------------------------------------------------------
\begin{frame}[fragile]{Atomic: Compare-and-swap}
  \begin{lstlisting}
  int atomic_cmpxchg (__global int *p, int cmp, int val)
  int atomic_cmpxchg (__local int *p, int cmp, int val)
  \end{lstlisting}

  Does:
  \begin{itemize}
    \item Read the 32-bit value (referred to as
    old) stored at location pointed by p.
  \item Compute \texttt{(old == cmp) ? val : old}.
  \item Store result at location pointed by p.
  \item Returns old.
  \end{itemize}
  \uncover<2>{
    \begin{tikzpicture} [overlay]
      \node [above left=1cm of current page.south east, draw,drop shadow,fill=white,
      inner xsep=0.5cm,inner ysep=0.5cm,thick]
        {
          Implement atomic \texttt{float} add?
        } ;
    \end{tikzpicture}
  }
\end{frame}
% }}}
% -----------------------------------------------------------------------------
\section{Intro to MPI}
% -----------------------------------------------------------------------------
% {{{
\begin{frame}{MPI}
  \textbf{M}essage \textbf{P}assing \textbf{I}nterface:
  \begin{center}
    \begin{tikzpicture}
      \node (node1) {
        \includegraphics[height=3cm]{server.jpeg}
      } ;
      \node [right=1cm of node1] (msg) {
        \includegraphics[height=2cm]{envelope.jpeg}
      } ;
      \node [right=1cm of msg] (node2) {
        \includegraphics[height=3cm]{server.jpeg}
      } ;
      \draw [dashed,thick,->] (node1) -- (msg) ;
      \draw [dashed,thick,->] (msg) -- (node2) ;
    \end{tikzpicture}
  \end{center}
\end{frame}
\begin{frame}{MPI}
  \foreach \i in {1,2,...,110}
  {%
    \ifthenelse{\equal{\i}{103}}{%
      \uncover<1-3>{
        \tikz \node (node\i) {\includegraphics[height=0.5cm]{server-small.jpeg}}; %
      }
    }{%
      \tikz \node (node\i) {\includegraphics[height=0.5cm]{server-small.jpeg}}; %
    }
  }
  \uncover<+>{}
  \only<+->{
  \foreach \i in {1,6,...,110}
  {
    \foreach \j in {3,23,...,110}
    {
      \tikz [overlay] \draw [thick,->] (node\i) -- (node\j) ;
    }
  }
  }

  \uncover<+->{
    \begin{tikzpicture} [overlay]
      \node [below right=1cm of current page.north west, draw,drop shadow,fill=white,
      inner xsep=0.5cm,inner ysep=0.5cm,thick]
        {
          Not enough throughput? Just buy more computers$^{*}$
        } ;
    \end{tikzpicture}
  }
  \uncover<+>{}
  \uncover<+->{
    \begin{tikzpicture} [overlay]
      \node [above left=0.5cm of current page.south east, draw,drop shadow,fill=white,
      inner xsep=0.5cm,inner ysep=0.5cm,thick,text width=0.6\textwidth]
        {
          Key questions:
          \begin{itemize}
            \item Who can I send mail to?
            \item How much mail can go through the system (bandwidth)?
            \item How fast does mail arrive (latency)?
            \item Should I wait for the return receipt?
            \only<+>{\item \textbf{Why haven't I heard from the other guys
            yet?}}
          \end{itemize}
        } ;
    \end{tikzpicture}
  }
\end{frame}
\addimgcredit{Server: sxc.hu/Kolobsek}
\addimgcredit{Envelope: sxc.hu/ilco}
% -----------------------------------------------------------------------------
\begin{frame}{MPI}
  \begin{columns}
    \column{0.5\textwidth}
      \begin{center}
      \textbf{MPI 3.0}

      \medskip
      Born September 21, 2012

      \vspace{2cm}
      \footnotesize
      MPI 1.0: June 1994
      \end{center}
    \column{0.5\textwidth}
      \includegraphics[width=\textwidth]{gift-box.png}
  \end{columns}
\end{frame}
\addimgcredit{Gift box: sxc.hu/iprole}
% -----------------------------------------------------------------------------
\begin{frame}{MPI}
  \begin{center}
  \Huge Demo time
  \end{center}
\end{frame}
% }}}


% Non-overtaking, order
% Progress

% Using barriers to split 'any' recvs.

% Collectives

% Communicators

% MPE, Jumpshot
% -----------------------------------------------------------------------------



\questionframe{}
\imagecreditslide

\end{document}
% vim: foldmethod=marker
